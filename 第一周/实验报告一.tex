\documentclass[a4paper,12pt]{article}
\usepackage{graphicx} % Required for inserting images
\usepackage{color}
\usepackage[UTF8]{ctex}
\usepackage{amsmath}
\usepackage{tabularray}



%\newcommand{\upsite}[1]{\textsuperscript}{cite{#1}}

%\bibliographystyle{unsrt}
\begin{document}
\begin{figure}[t]
    \includegraphics[width=0.2\textwidth]{ouc.jpg}
\end{figure}

\title{实验报告}
\author{单衍喆 }
\date{\today}
\maketitle

\pagenumbering{roman}
\text{GitHub地址:https://github.com/Venusss1/course.git}
\tableofcontents
\newpage
%\pagenumbering{arabic}


\section{\underline{\color{blue}实验内容}}

\begin{enumerate}
    \item \textbf{LaTex文档编辑}
    \item \textbf{版本控制工具git}
\end{enumerate}
\section{\underline{\color{blue}实验设计}}
\subsection{\color{red}LaTex文档编辑}

\subsubsection{\color{green}LaTex安装与环境配置}
\begin{enumerate}
    \item 进入网站清华大学开源软件站下载TexLive
          \begin{figure}[htbp]
              \centering
              \includegraphics[width=0.5\textwidth]{website.png}
              \caption{清华大学开源软件站}
              \label{website}
          \end{figure}
    \item 执行ios文件安装
    \item vscode配置环境
          \begin{itemize}
              \item[+] 安装LaTex workshop 插件
              \item[+] 配置.json文件
          \end{itemize}
\end{enumerate}
\subsubsection{\color{green}基本命令}

\begin{table}[htbp]
    \centering
    \caption{\color{green}基本命令}
    \begin{tblr}{
            cell{2}{1} = {c,fg=cyan},
            cell{3}{1} = {c,fg=cyan},
            cell{4}{1} = {c,fg=cyan},
            cell{5}{1} = {c,fg=cyan},
            cell{6}{1} = {c,fg=cyan},
            vline{2} = {1-5}{},
            hline{1-2} = {-}{},
        }
        符号 & \textsf{作用}      \\
        \% & \textsf{注释}      \\
        \textbackslash{}  & \textsf{命令和特殊意义} \\
        \$ & \textsf{公式}      \\
        \&   & \textsf{对齐}
    \end{tblr}
\end{table}

\subsubsection{\color{green}文档结构}
\begin{enumerate}
    \item 设定区域
          \begin{itemize}
              \item documentclass{}
              \item usepackage{}
          \end{itemize}
    \item 各级标题
          \begin{itemize}
              \item cheaper -- 章
              \item section -- 节
              \item subsection -- 小节
              \item subsubsection -- 小小节
          \end{itemize}
    \item 段落设置
          \begin{itemize}
              \item par:分段
              \item newpage:分页命令
          \end{itemize}
\end{enumerate}
\subsubsection{\color{green}插入}
\begin{itemize}
    \item 数学公式
          \begin{itemize}
              \item 正文行中的特殊字符和短公式:使用两个\$包括要表达的公式
              \item 单行公式带编号:equation
              \item 无编号公式:使用双\$包括
              \item 多行公式:split(usepackage(amsmath))
              \item 多情况讨论:cases(usepackage(amsmath))
              \begin{equation}
                F(x)=
                \begin{cases}
                    0&,\color{blue}\text{if $x=0$}\\
                    -x+1&, \color{blue}\text{if $x>0$}\\
                    x+1&, \color{blue}\text{if $x<0$}
                \end{cases}
              \end{equation}
          \end{itemize}
    \item 图片
          \begin{itemize}
              \item usepackage\{graphicx\}
              \item begin\{figure\}...end\{figure\}\\begin\{figure*\}...end\{figure*\}
              \item 常用命令:
                    \begin{itemize}
                        \item centering:居中
                        \item includegraphics[图片大小][图片路径]
                        \item caption\{图片说明\}
                        \item label\{标签\}
                    \end{itemize}
              \item htbp
                    \begin{itemize}
                        \item h(here):尽量放置在代码所在位置
                        \item t(top):放置在页面顶部
                        \item b(bottom):放置在页面底部
                        \item p(page):单独放置在一个页面
                    \end{itemize}
          \end{itemize}
    \item 表格\\选择在线生成工具:latex-tables.com
    \item 文献引用
          \begin{itemize}
              \item cite\{lable\}
              \item begin\{thebibliography\}\\bibitem\{lable1\}...\\bibitem\{lable2\}\\end\{thebibliography\}
              \item BIBTex管理文献
                    \begin{enumerate}
                        \item 设定区域:bibliographystyle\{unsrt\}
                        \item bib文件:@article\{title,...\}
                        \item 插入位置:cite\{title\}
                        \item 参考文献位置:bibliography\{bib文件名\}
                    \end{enumerate}
          \end{itemize}
\end{itemize}

\subsection{\color{red}git版本控制}
\subsubsection{\color{green}git安装与环境配置}
\begin{enumerate}
    \item 安装git
          \begin{figure}[htbp]
              \centering
              \includegraphics[width=0.5\textwidth]{gitt.png}
              \label{git}
              \caption{git官网}
          \end{figure}
    \item git -v 检查版本信息
    \item 配置用户名和邮箱\\git config --global user.name "venus"\\git config -- global user.email 3228693652@qq.com

\end{enumerate}
\subsubsection{\color{green}基础命令}
\begin{itemize}
    \item \textsf{repository(仓库)}
          \begin{itemize}
              \item git init 新建仓库
              \item git clone 复制仓库
              \item git status 查看仓库状态
          \end{itemize}
    \item \textsf{提交}
          git add %支持文件夹 
          添加到暂存区\\git commit (-a)(-m text)提交%支持通配
          \\git log (--graph)(--oneline)查看提交记录
    \item \textsf{回退版本}
          \begin{itemize}
              \item git reset --soft
              \item git reset --hard
              \item git reset --mixed
              \item git diff 查看工作区、暂存区、本地仓库的区别
          \end{itemize}
    \item \textsf{比較}
          \begin{itemize}
              \item git diff 比較工作區和暫存區
              \item git diff HEAD 工作區+暫存區 比較 本地倉庫
              \item git diff -staged 暂存区和本地仓库
                    %可比较提交
          \end{itemize}
    \item \textsf{删除}
          \begin{itemize}
              \item rm file
              \item git rm <file> 从工作区和暂存区同时删除\cite{label1}
              \item git rm --cached <file> 从暂存区删除\cite{label2}
              \item git rm -r* 递归删除所有子目录和文件
          \end{itemize}
    \item \textsf{忽略(.gitignore)}
          \begin{itemize}
              \item[-] 系统或软件自动生成
              \item[-] 编译产生的文件
              \item[-] 日志文件,缓存文件,临时文件
              \item[-] 身份、密码等敏感信息
          \end{itemize}
\end{itemize}
\subsubsection{\color{green}分支}
\begin{itemize}
    \item git branch 创建分支
    \item git switch切换分支
    \item git merge (--abort)合并分支
    \item git branch -d 删除分支
\end{itemize}
\subsubsection{\color{green}别名}
alias graph="git log --oneline --graph --decorate --all"

\newpage

\begin{thebibliography}{99}
    \bibitem{label1}Anita,ekjfisfjs
    \bibitem{label2}Bieat,sfsdfwegbraref
\end{thebibliography}





\end{document}
